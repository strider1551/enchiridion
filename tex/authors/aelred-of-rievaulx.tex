\section*{Abbreviations}
\begin{tabular}{ l l }
  SF & Spiritual Friendship \\
\end{tabular}\bigskip

\noindent N.B.: Citations make use of the section numbering used by Lawrence Braceland in the 2010 Cistercian Publication's edition (ISBN 978-0-87907-970-3).

\section*{General Concepts}

\begin{outline}
    \one Friendship in General & \\
        \two Friendship must begin in Christ, continue with Christ, and be perfected by Christ & 1.10 \\
            \thr Borrowing from Cicero, friendship is agreement in things human and divine, with good will and charity & 1.11 \\
            \thr A friend is the guardian of the soul itself. & 1.20 \\
        \two Love is an attachment of the rational soul. Through love, a soul seeks an object and enjoys it with interior sweetness. & 1.19 \\
            \thr In friendship, the spirits of two people are united in sweetness. & 1.21 \\
        \two A friend loves always. & 1.24 \\
            \thr Quoting St. Jerome, a friendship that can end was never true. & 1.24 \\
        \two Many more are embraced by charity than by friendship. Only those to whom we freely entrust the entirety of our hearts are friends. 1.32 \\
        \two Friendship is part of human nature, but is developed through life experience and sanctified by God's law. & 1.51 \\
    \one Categories of Friendship & \\
        \two Carnal friendship & \\
            \thr Comes from the desire for pleasure and self-fulfillment.\footnote{Braceland's translation could suggest a sexual nature to this friendship, but I do not think it should be read exclusively as such.} & 1.39 \\
            \thr These friends are willing to break moral boundaries for each other. & 1.40 \\
            \thr Dominated by affection, not moderation or reason. & 1.41 \\
        \two Worldy friendship & \\
            \thr Comes from a greed for temporal goods and wealth. & 1.42 \\
            \thr Endures only so long as their is good fortune. & 1.42-43 \\
        \two Spiritual friendship & \\
            \thr Comes from a likeness of life, habits, and interests. & 1.46 \\
            \thr Guided by prudence, ruled by justice, protected by strength, and moderated by temperance. & 1.49 \\
        \two Aelred's work is only concerned with the category of \textit{spiritual} friendships. Anywhere friendship is mentioned in general --- including the his remarks on friendship in general prior to categorizing them --- it should be first assumed that he is commenting on spiritual friendship. For example, he quotes St. Jerome in his general description of friendship and states that a friendship that can end was never a friendship, but then he says that wordly friendships end when good fortune subsides; it is only a true spiritual friendship that is incapable of ending. & \\
\end{outline}