\section*{Abbreviations}
\begin{tabular}{ l l }
  AC & The Ascent of Mount Carmel \\
  DN & The Dark Night \\
  FL & The Living Flame of Love \\
  LL & The Sayings of Light and Love \\
  SC & The Spiritual Canticle \\
\end{tabular}\bigskip

\noindent N.B.: Citations make use of the section numbering used by Kavanaugh and Rodriguez in their 1991 translation (ISBN 0-935216-14-6).

\section*{General Concepts}

\begin{outline}
	\one Appetites & \\
		\two Appetites are inordinate desires or longings of some sort. They are inordinate in that they do not lead to a spiritual good or do not hold that good as their final objective. & AC 1.1.1 \\
			\thr Involuntary appetites cannot be controlled by the will. They are natural, are not pursued beyond the first thought, and are not given consent. & AC 1.13.6 \\
			\thr Voluntary appetites are appetites that can be controlled by the will.  This includes appetites that began as involuntary but were then pursued by the will. & \\
			\thr John normally disregards involuntary appetites.  Unless specified otherwise, "appetites" is a restricted term that only refers to voluntary appetites. & \\
		\two Virtues are the opposites of appetites and have the opposite effects (see \ref{appetites_effects}). & AC 1.12.5 \\
		\two The mortification of one's appetites, which is depriving oneself of taking gratification in them, is necessary to attain Divine union. & AC 1.4.1 \\
			\thr All appetites must be mortified, not just some. & AC 1.11.2 \\
		\two \label{appetites_effects}Appetites cause harmful effects in the soul.  Some effects are privative (take something) and others are positive (add something). & AC 1.6.1 \\
			\thr \textit{privative}: They deprive the soul of God's Spirit. & \\
				\for A complete loss of God's Spirit is caused only by voluntary appetites involving mortal sin. & AC 1.12.3 \\
			\thr \textit{positive}: They weary, torment, darken, defile, and weaken. & \\
				\for Each appetite causes all of these effects.  Only voluntary appetites involving mortal sin cause the effects totally. & AC 1.12.3 \\
				\for Only one positive effect will be caused principally and directly, the others will be caused indirectly. & AC 1.12.4 \\
			\thr Appetites are not equally detrimental in their effects.  An individual may suffer greater harm from one of his or her appetites than from another. & AC 1.11.2 \\
	\one The Dark Nights & \\
		\two The dark nights are periods of purification.  There are two principle nights: the Night of Sense and the Night of Spirit. Both have two primary subdivisions: the active night and the passive night. & \\
			\thr \textit{active}: The individual is actively doing things to purify his or her self. & \\
			\thr \textit{passive}: The individual is passive while receiving the work of God in his or her soul. & \\
			\thr The passive nights alone are sometimes equated with the dark nights, relegating the active nights to the activity one does to enter into the nights.  I suggest that if the focus is on the purification of the soul, then both active and passive nights are proper to the dark nights; if the focus is on the perceived experience of the night, then only the passive nights are proper to the dark nights while the active nights are prefaces. & \\
			\thr While the active nights precede the passive nights, it must be stressed that the activity of the individual does not cause the passive night to take place.  The passive nights are entirely the action of God, and He grants those nights to only those He desires to. & \\
		\two The Dark Night of Sense & \\
			\thr This pertains to the sensory part of the soul and consists of the process of the mortification of appetites. & AC 2.2.2, 1.12.1  \\
			\thr This night is more a reformation of appetites than a true purgation.  All sensory imperfections are rooted in the spirit, and so a sensory purgation is not complete until the spirit is also purged. & DN 2.3.1 \\
		\two The Dark Night of Spirit & \\
			\thr This pertains to the rational part of the soul and consists of the reformation of intellect, memory, and will. & AC 2.2.2 \\
		\two The nights are not the same for everyone.  They can vary in length of time and intensity according to the person and the degree of love God wishes to raise the soul. & DN 1.14.5 \\
\end{outline}

\section*{The Path of Spiritual Progression}

\begin{outline}
	\one Beginners, the Period Before the Night of Sense &  \\
		\two Beginners are motivated by the consolation and satisfaction of spiritual works, not the spiritual works themselves. & DN 1.1.3 \\
		\two Beginners want God to desire what they want, and become sad if they have to desire God's will. & DN 1.7.3 \\
		\two Imperfections of pride & DN 1.2 \\
			\thr They have a vain pride in their spiritual practices and seek glory from them.  They sometimes desire to instruct rather than be instructed.  They may internally condemn those who do not have their devotions, or even externally condemn and detract other to appear holy by comparison. &  \\
			\thr They seek recognition for their spiritual practices.  If their spiritual director disapproves of them they feel the director does not understand and may seek a new one who will be impressed by them. &  \\
			\thr They avoid a loss of spiritual reputation.  They confess sins in a favorable light, minimize their faults, or may go to a different confessor prior to their regular confessor. &  \\
		\two Imperfections of avarice & DN 1.3 \\
			\thr They are unhappy, not finding the consolation they want from their spiritual practices.  They seek new counsels, maxims, and books trying to find that consolation. &  \\
			\thr They become possessive and attached to the elaborateness and decoration of devotional objects like crosses and rosaries. &  \\
		\two Imperfections of lust & DN 1.4 \\
			\thr They may experience impure movements in the sensory part of the soul proceeding from spiritual things. &  \\
		\two Imperfections of anger & DN 1.5 \\
			\thr They become easily angered by little things after the delight and satisfaction of their spiritual exercises are gone. &  \\
			\thr They become angry over the sins of others, possibly reproving them in anger. &  \\
			\thr They grow angry with themselves and their imperfections, make great plans in their impatience while waiting for God to provide what they need, and become more angry when those fail. &  \\
				\for Others are so patient for advancement that God would prefer it a little less. &  \\
		\two Imperfections of gluttony & DN 1.6 \\
			\thr They strive more for spiritual favor and delight than for spiritual purity.  Their prayer consists in seeking sensory satisfaction. &  \\
			\thr They confuse gratifying and satisfying themselves with serving and satisfying God.  They think they cannot serve God if not doing what they want, and may hide penances from a spiritual director who does not allow them to do as they wish. &  \\
			\thr They have an aversion to self-denial. &  \\
		\two Imperfections of envy & DN 1.7 \\
			\thr They feel sad about the spiritual perfection of others, and may try to contradict praise given to people other than themselves. &  \\
		\two Imperfections of sloth & DN 1.7 \\
			\thr They flee from spiritual exercises that do not give sensory satisfaction, becoming bored and either avoiding such practices or doing them begrudgingly. &  \\
    \one The Active Night of Sense &  \\
    	\two Conduct of those in this night & AC 1.13 \\
    		\thr Have a habitual desire to imitate Christ. & AC 1.13.2 \\
    			\for Imitating and behaving as Christ requires studying His life. &  \\
    		\thr Renounce sensory satisfaction that is not for the glory of God.  Do not desire to hear things or look upon things or speak about things, and so on, unless it helps you to love God more. & AC 1.13.3 \\
    			\for This should be done out of love for Christ. &  \\
    			\for If satisfaction cannot be avoided, it is sufficient to have no desire for that satisfaction. &  \\
    		\thr Endeavor to be inclined to the difficult, unpleasant, unconsoling instead of the easy, pleasant, consoling, and so on.  That is, embrace the worst in temporal things instead of the best. & AC 1.13.6 \\
    			\for This is a habit of mind and an interior attitude. &  \\
    		\thr Act, speak, and think contemptuously of yourself and desire the same from others. & AC 1.13.9 \\
    	\two Those who are sincere in the practices of the active night of sense, doing them with order and discretion, will find delight and consolation in them. & AC 1.13.7 \\
    \one The Passive Night of Sense &  \\
    	\two There are three signs indicating someone is experience the passive night of sense.  All of them must be present simultaneously. & DN 1.9 \\
    		\thr They find no delight in anything, God or creatures. &  \\
    		\thr They seek God solicitously, but believe they are not serving him because of the lack of delight. &  \\
    		\thr They cannot meditate and make use of the imagination in prayer as they used to. &  \\
    	\two Conduct of those in this night & DN 1.10 \\
    		\thr They should not attempt discursive meditation, but instead remain in restful, loving attention to God, even though it seems they are doing nothing. &  \\
    		\thr They should be without concern, effort, or desire for sensory satisfaction in God. &  \\
    			\for The more they seek support in knowledge and affection through sensory means, the more their soul will feel the lack of such things. &  \\
    	\two Benefits caused by this night & DN 1.12-13 \\
    		\thr Knowledge of one's self and one's own misery. &  \\
    		\thr Knowledge of God's grandeur and majesty. &  \\
    		\thr Regarding the spiritual vices of beginners: &  \\
    			\for A humility removes spiritual pride.  The thought of being more advanced does not occur as they realize others are more mature.  From this stems a love of others and a willingness to listen and be directed. &  \\
    			\for A detachment from spiritual avarice gives a readiness for spiritual exercises without gratification. &  \\
    			\for The soul is freed from spiritual lust. &  \\
    			\for Spiritual glutony is purged by the mortification of appetites. &  \\
    		\thr A habitual remembrance of God with a fear of turning back on the spiritual road. &  \\
    		\thr The soul exercises all the virtues together. &  \\
    		\thr Spiritual angry, envy, and sloth are all purged as the soul acquires the opposite virtues. &  \\
    		\thr No longer acting out of delight and satisfaction in a work, solicitude for God and longings about serving him increase. &  \\
    \one Proficients, the Period Before the Night of Spirit &  \\
    	\two Proficients experience more freedom, satisfaction of spirit, and interior delight than they did prior to the night of sense. & DN 2.1.1 \\
    	\two They may experience intermittent periods of need, aridity, darkness, and conflict that are purgative but are not as intense or as lengthy as the night of spirit. & DN 2.1.1 \\
    	\two They still suffer from habitual and actual imperfections. & DN 2.2 \\
    		\thr Notably, spiritual communications, apprehensions, and feelings may become tools of the devil if the proficient is not vigilant in renouncing them. &  \\
    \one The Active Night of Spirit &  \\
    	\two Detach from understanding, feeling, imagining, opinion, desire, and one's own way. & AC 2.4.4 \\
    	\two The three faculties (intellect, memory, and will) must be emptied through faith, hope, and charity, respectively. & AC 2.5.6 \\
    	\two Purification of Intellect through Faith: AC 2.7 - 2.32 &  \\
    		\thr Signs to move from meditation to contemplation.  All must be present simultaneously. & AC 2.13.5-6 \\
    			\for Cannot meditate or receive the same satisfaction as before. & AC 2.13.2 \\
    			\for Disinclination to focus imaginations. & AC 2.13.3 \\
    			\for Prefers to remain alone in loving awareness of God without exercise of intellect, memory, or will. & AC 2.13.4 \\
    		\thr The transition to contemplation is not immediate, and those in transition will sometimes meditate and sometimes contemplate. & AC 2.15.1 \\
    	\two Purification of Memory through Hope: AC 3.1 - 3.15 &  \\
    		\thr Try to forget sensory experience. & AC 3.2.14 \\
    		\thr Failing to darken the memory from natural knowledge and discursive reflection results in three types of harm. & AC 3.3.1 \\
    			\for Subjection to evils arising from such knowledge: falsehoods, imperfections, appetites, judgments, and loss of time. & AC 3.3.2 \\
    			\for Subjection to demonic influence in the form of knowledge and deluge that moves one towards harmful emotions and vices. & AC 3.4.1 \\
    			\for Removing tranquility and peace from the soul by unsettling it through encumbered memories. & AC 3.5.1 \\
    		\thr Failing to darken the memory from supernatural imaginative knowledge results in five types of harm. & AC 3.8.1 \\
    			\for Will often be deluded and mistake the natural for the supernatural. & AC 3.8.2 \\
    			\for Susceptible to presumption and vanity. & AC 3.9.1 \\
    			\for Devil is able to deceive through false knowledge. & AC 3.10.1 \\
    			\for Impedes union with God in hope. & AC 3.11.1 \\
    			\for Judges God in a lowly way. & AC 3.12.1 \\
    		\thr Spiritual knowledge of creatures may be remembered so long as it produces a good effect.  Spiritual knowledge of the Creator should be remembered as often as possible. & AC 3.14.2 \\
     	\two Purification of Will through Charity: AC 3.16 - 3.45 &  \\
    		\thr The will may possess four inordinate emotions that give rise to appetites and affections: joy, hope, sorrow, and fear.  Through purification they can become properly ordered such that one rejoices only in God, hopes only in God, and has sorrow and fear only when it pertains to such things.  John does not fulfill his intention to write on all four emotions, with the text ending part way through his discussion of the first emotion, joy. & AC 3.16.1 \\
    		\thr Joy is the delight of the will in an object esteemed and considered fitting.  John's discussion of joy is limited to joy in an active sense, when a person understands the object of his or her joy and actively decided to rejoice in it. & AC 3.17.1 \\
    		\thr Joy derives from six objects: temporal, natural, sensory, moral, supernatural, and spiritual.  Harm results from desiring joy through any of these goods, but great benefits result from restraining one's desire and detaching from them. & AC 3.17.2 \\
    			\for These goods produce four distinct types of joy: motivating, provocative, directive, and perfective.  Again, harm results from becoming attached to the experience of joy and not the One whom joy leads to. & AC 3.35.1 \\        
    \one The Passive Night of Spirit &  \\
     	\two God strips the faculties, affections, and senses.  The intellect is left in darkness, the will in aridity, the memory in emptiness, the affections in bitterness and anguish, and the senses from satisfaction. & DN 2.4.3 \\
     	\two The night does not give pain, but sweetness. However, the weakness and imperfection of the soul, its inadequate preparation, and its qualities contrary to the night causes suffering as divine light shines upon it. Afflictions suffered in the night: & DN 2.9.11 \\
    		\thr The divine light reveals the impurity of the individual so clearly to him or her that it seems God is against them and them against God. The soul understands it is worthy of neither God nor creatures, thinks it will never be worthy, and thinks that it will receive no more blessings. & DN 2.5.5 \\
    		\thr Divine contemplation forcibly subdues their natural, moral, and spiritual weakness. They would consider death a relief and believe no one will take pity on them. & DN 2.5.6 \\
    		\thr Purgations of the soul feels like a spiritual death, the sorrows of hell, and rejection by God. & DN 2.6.1-2 \\
    		\thr They feel forsaken and despised by all creatures, particularly their friends. & DN 2.6.3 \\
    		\thr They cannot beseech God or raise their mind to Him. & DN 2.8.1 \\
     	\two Those in this night deserve deep compassion because of their immense suffering. Still, they are not able to receive consolation or support from any doctrine or spiritual master. & DN 2.7.3 \\
     	\two It may last for years, although there may be intervals when God temporarily lifts the night. & DN 2.7.4 \\
\end{outline}